\section{INTRODUCTION}

Reinforcement learning (RL) learns a policy that maximizes rewards through trial and error.
Learning in this manner may look very simple and straightforward, yet it is highly effective.
This effectiveness has been clearly demonstrated in practice.
Over the past few years, RL has demonstrated impressive achievements in diverse applications \cite{silver2017mastering}, \cite{andrychowicz2020learning}, \cite{schrittwieser2020mastering}, \cite{ouyang2022training}.
Nevertheless, deploying RL in physical real-world environments remains a major challenge.
To deploy RL-trained agents in real-world environments, two requirements must be satisfied:
\textcolor{red}{First}, the agent must be able to successfully accomplish the given tasks, and second, it must ensure safety and reliability.
In standard RL, such requirements are learned exclusively from reward signals, thereby necessitating careful reward engineering.
However, reward engineering is inherently difficult, and even with carefully designed rewards, the learned policy may exploit unintended loopholes (reward hacking) or remain hard to interpret, making it difficult to ensure safety and reliability \cite{amodei2016concrete}.
These limitations highlight the necessity of an explicit constraint function, separated from the reward, which has motivated research on constrained reinforcement learning (constrained RL).

Constrained RL is a method that learns a policy to maximize rewards while satisfying constraints, thus enabling agents to behave safely while successfully performing tasks.
A common formulation of constrained RL specifies constraints at the trajectory level \cite{brunke2022safe}.
While such a formulation ensures that the overall constraint is satisfied, violations can still occur at individual states.
Importantly, even a single violation may sometimes lead to severe consequences.
For example, while a trajectory-level constraint might successfully restrict the total number of lane departures during a trip, a single deviation at a critical moment---such as near another vehicle---can nevertheless lead to catastrophic failures.
This limitation highlights the need to consider state-wise constrained RL, which explicitly enforces constraints at the level of individual states.
Accordingly, we extend the Lagrangian-based approach, a representative method in constrained RL, by introducing state-wise Lagrange multipliers, encouraging the policy to satisfy safety requirements at every timestep.Our contributions are as follows:
\begin{itemize}
    \item[$\bullet$] We leverage neural networks to estimate state-wise Lagrange multipliers, enabling the policy to satisfy state-wise constraints more effectively.
    \item[$\bullet$] We demonstrate that the proposed method allows for finer specification of constraints and achieves better constraint satisfaction compared to existing constrained RL approaches under the same training budget.
    \item[$\bullet$] We show that the learned Lagrange multiplier network can be utilized at deployment time to assess state-wise safety, providing interpretability and insight into the agent’s behavior.
\end{itemize}